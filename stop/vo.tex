\vprasanje{Definiraj topologijo na $X$.}

Topologija na $X$ je družina $\T \subseteq X$ podmnožic, za katero velja:
\begin{itemize}
  \item $\varnothing, X \in \T$
  \item $\{U_\lambda\}_{\lambda \in \Lambda} \subseteq \T
	\implies \bigcup_{\lambda \in \Lambda} \in \T$
  \item $\{ \nastevanje{U}{,}{n} \} \subseteq \T
	\implies \nastevanje{U}{\cap}{n} \in \T$
\end{itemize}

\vprasanje{Kaj je okolica točke $x \in X$?}

To je vsaka taka odprta množica $U \in \T$, ki vsebuje $x$.

\vprasanje{Kdaj je točka $x \in A$ notranja?}

Kadar $A$ vsebuje neko okolico $x$.

\vprasanje{Kdaj je točka $x \in X$ mejna točka podmnožice
  $A \subseteq X$?}

Kadar vsaka okolica $x$ seka tako $A$ kot $A\complement$.

\vprasanje{Kaj je notranjost, meja in kaj zaprtje množice
  $A \subseteq X$?}

$\Int A = \text{Največja odprta podmnožica $A$} =
\bigcup_{U \in \T, U \subseteq A} U$.

$\Cl A = \text{Najmanjša zaprta nadmnožica $A$} =
\bigcap_{A \subseteq U \in T} U$

$\Fr A = \Cl A \brez \Int A$.

\vprasanje{Kdaj je $x \in X$ stekališče $A \subseteq X$?}

Kadar vsaka okolica $x$ seka $A \brez \{x\}$.

\vprasanje{Kdaj je $x \in X$ limita zaporedja $(x_n)_n$?}

Kadar vsaka okolica $x$ vsebuje rep zaporedja.

\vprasanje{Definiraj in karakteriziraj zveznost funkcije $f:X \to Y$.}

$f$ je zvezna, če je praslika vsake odprte množice v $Y$ odprta v $X$.

Ekvivalentno, če je praslika vsake zaprte množice v $Y$ zaprta v $X$.

Ekvivalentno, če je $\forall A \subseteq X . f(\cl{A}) \subseteq \cl{f(A)}$.

\vprasanje{Kdaj je $f: X \to Y$ odprta in kdaj zaprta? Kaj velja, če je
  $f$ bijekcija?}

Odprta je, če odprte množice slika v odprte.

Zaprta je, če zaprte množice slika v zaprte.

Če je $f$ bijektivna, je odprta natanko tedaj, ko je zaprta.

\vprasanje{Definiraj in karakteriziraj homeomorfizem.}

To je bijekcija $f:X \to Y$, ki je zvezna in ima zvezen inverz.

Ekvivalentno, to je bijekcija $f: X \to Y$, ki je zvezna in odprta.

Ekvivalentno, to je bijekcija $f: X \to Y$, ki je zvezna in zaprta.

\vprasanje{Kaj je topološka lastnost?}

To je lastnost prostora, ki se ohranja pri homeomorfizmih.

\vprasanje{Definiraj bazo prostora $X$.}

To je množica $B \subseteq \T$, da velja $\T = \{ \text{unije eltov $B$} \}$.

\vprasanje{Naj bo $B \subseteq P(X)$. Kdaj je množica
  $\T = \{ \text{unije eltov $B$} \}$ topologija na $X$?}

Natanko tedaj, ko je $\forall S, S' \in B . S \cap S' =
\{ \text{unija nekih eltov $B$} \}$, in $\bigcup B = X$.

\vprasanje{Kako določiš zaprtje in notranjost množice s pomočjo baze?}

V vsaki točki $x$ iz zaprtja vsaka bazna okolica seka množico.

Za vsako točko $x$ iz notranjosti obstaja bazna okolica, ki je povsem v množici.

\vprasanje{Definiraj produktno topologijo.}

Naj bosta $X$ in $Y$ topološka prostora, ter $B_X$ in $B_Y$ njuni bazi.
Tedaj je $B_x \times B_Y$ baza za topologijo na $X \times Y$.

\vprasanje{Kakšni preslikavi sta $\pr_X : X \times Y \to X$
  in $\pr_Y : X \times Y \to Y$?}

Zvezni in odprti, nista pa zaprti.

\vprasanje{Kdaj je $f = (f_1, f_2) : X \to Y_1 \times Y_2$ zvezna?}

Natanko tedaj, ko sta $f_1$ in $f_2$ zvezni.

\vprasanje{Kaj je predbaza topologije?}

Množica $P \subseteq P(X)$, če je $B = \{ \text{končni preseki eltov $P$} \}$
baza.

\vprasanje{Kako preveriš, da je $P$ predbaza $\T$?}

Veljati mora $P \subseteq \T$ in $\T \subseteq \left<P\right>$.

\vprasanje{Ali lahko zveznost preverjaš le na baznih eltih?
  Kaj pa le na predbaznih?}

Da.

\vprasanje{Kaj je lokalna baza okolic za točko $x \in X$?}

Taka družina odprtih množic, da vsaka okolica $x$ vsebuje nek elt lokalne baze.

\vprasanje{Kdaj je prostor 1-števen in kdaj 2-števen?
  Kako sta lastnosti povezani?}

Prostor $X$ je 1-števen, če za vsako točko $x \in X$ obstaja
števna lokalna baza okolic.

Prostor $X$ je 2-števen, če obstaja števna baza.

Če je prostor 2-števen, je 1-števen.

\vprasanje{Kako dobimo bazo topologije, če imamo dane vse lokalne baze?}

Unija lokalnih baz je baza topologije.

\vprasanje{Kaj je $L(A)$? Kakšen je v primerjavi z $A$ in z $\cl{A}$?}

To je množica vseh limit zaporedij v $A$.
Velja $A \subseteq L(A) \subseteq \cl{A}.$

\vprasanje{Kdaj je $A \subseteq X$ povsod gosta? Povej dve karakterizaciji.}

Kadar je $\cl{A} = X$.

To je natanko tedaj, ko $A$ seka vsako odprto množico v $X$.

\vprasanje{Kdaj je $X$ separabilen?}

Kadar obstaja števna gosta podmnožica.

\vprasanje{Kaj velja za števnostne lastnosti metričnih prostorov?}

Vsi metrični prostori so 1-števni.

Metrični prostori so separabilni natanko tedaj, ko so 2-števni.

\vprasanje{Definiraj topološki podprostor.}

Naj bo $A \subseteq X$. Opremimo ga z inducirano topologijo
$\T_A = \{ U \cap A \such U \in \T_X \}$.

\vprasanje{Opiši zaprte množice v podprostoru $A \subseteq X$.}

To so natanko preseki zaprtih množic v $X$ z $A$.

\vprasanje{Kako dobiš bazo za podprostor $A \subseteq X$?}

Sekaš elte baze za $X$ z $A$.

\vprasanje{Naj bo $B \subseteq A \subseteq X$. Povej razmerja med zaprtjem,
notranjostjo ter mejo $B$ v $A$ ter v $X$.}

$\Cl_A B = (\Cl_X B) \cap A$

$\Int_A B \supseteq \Int_X B$

$\Fr_A B \subseteq \Fr_X B$

\vprasanje{Naj bo $B \subseteq A$ odprta množica. Kaj mora veljati, da bo
$B$ sigurno odprta tudi v $X$?}

$A$ mora biti odprta v $X$.

\vprasanje{Naj bo $f: X \to Y$ zvezna in $A \subseteq X$.
  Kaj mora veljati za $A$, da bo zožitev $f$ na $A$ zvezna?}

Vedno bo zvezna.

\vprasanje{Kdaj je zaprto pokritje $\{X_\lambda\}_\lambda$ lokalno končno?}

Kadar za vsak $x \in X$ obstaja okolica $U \ni x$, ki seka natanko končno mnogo
$X_\lambda$.

\vprasanje{Kako preverjaš zveznost funkcije, definirane kosoma na pokritju?}

Če je pokritje zaprto lokalno končno ali odprto, in so vse zožitve zvezne,
in je $f$ enolična (se ujema na presekih pokritja), potem je $f$ zvezna,
in je edina zvezna funkcija s takimi zožitvami.

\vprasanje{Kaj je vložitev?}

To je funkcija $f: X \to Y$, ki je homeomorfizem na svojo sliko z inducirano
topologijo na sliki.

\vprasanje{Kateri razred preslikav v metričnih prostorih so
avtomatično vložitve?}

Izometrije.

\vprasanje{Kaj velja za odprte in zaprte vložitve?}

Če je $f: X \to Y$ zvezna injekcija in $f(X)$ odprta v $Y$, je $f$ vložitev
natanko tedaj, ko je odprta.

Če je $f: X \to Y$ zvezna injekcija in $f(X)$ zaprta v $Y$, je $f$ vložitev
natanko tedaj, ko je zaprta.

\vprasanje{Definiraj $\Ti{0}, \Ti{1}, \Ti{2}, \Ti{3}, \Ti{4}$.}

\begin{itemize}
  \item $\Ti{0}$: Za $x \ne y$ obstaja okolica $U \in \T$, da velja
  $x \in U \land y \notin U$ ali $x \notin U \land y \in U$.
  \item $\Ti{1}$: Topologija loči točke.
  \item $\Ti{2}$: Topologija ostro loči točke.
  \item $\Ti{3}$: Topologija ostro loči točke od zaprtih množic.
  \item $\Ti{4}$: Topologija ostro loči zaprte množice.
\end{itemize}

\vprasanje{Katere aksiome ločljivosti lahko preverjaš na baznih eltih?}

$\Ti{2}, \Ti{1}, \Ti{0}$ ter $\Ti{3}$ za bazne okolice točke.

\vprasanje{Kako drugače rečemo prostorom $\Ti{0}, \Ti{1}, \Ti{2}$?
Kdaj je prostor regularen in kdaj normalen?}

\begin{itemize}
  \item $\Ti{0}$: Kolmogorov
  \item $\Ti{0} \land \Ti{1}$: Fréchetov (dovolj $\Ti{1}$)
  \item $\Ti{0} \land \Ti{1} \land \Ti{2}$: Hausdorffov (dovolj $\Ti{2}$)
  \item $\Ti{0} \land \Ti{1} \land \Ti{2} \land \Ti{3}$:
    regularen (dovolj $\Ti{3} \land \Ti{0}$)
  \item $\Ti{0} \land \Ti{1} \land \Ti{2} \land \Ti{3} \land \Ti{4}$:
    normalen (dovolj $\Ti{4} \land \Ti{1}$)
\end{itemize}

\vprasanje{Katera družina prostorov je avtomatično normalna?}

Metrizabilni prostori.

\vprasanje{Karakteriziraj Hausdorffovo lastnost.}

$X$ je Hausdorffov natanko tedaj, ko za vsaki točki $x \ne y$ obstaja okolica
$U \ni x$, da velja $y \notin \cl{U}$.

Ekvivalentno, če je $\Delta_X$ zaprta v $X \times X$.

\vprasanje{Kaj pove Hausdorffova lastnost o zaporedjih in o zveznih funkcijah?}

Če je $Y$ Hausdorffov, imajo zaporedja v $Y$ največ eno limito.

Če sta $f, g: X \to Y$ zvezni, je množica rešitev enačbe $f = g$ zaprta v $X$.

Če se dodatno ujemata na gosti podmnožici, velja $f = g$ povsod na $X$.

Če je $f: X \to Y$ zvezna, je njen graf zaprta podmnožica v $X \times Y$.

\vprasanje{Kaj mora veljati, da lahko zamenjaš limito in funkcijo v zapisu
$f(\lim_i x_i)$?}

$f: X \to Y$ mora biti zvezna, zaporedje $(x_i)_i$ konvergentno, $X$ 1-števen
in $Y$ Hausdorffov.

\vprasanje{Kakšna je povezava med Fréchetovim prostorom in topologijo končnih
komplementov?}

Če je $X$ Fréchetov, velja $\T_X \supseteq \T_{KK}$.

\vprasanje{Karakteriziraj Fréchetovo in Hausdorffovo lastnost z enojci.}

$X$ je Fréchetov natanko tedaj, ko je za vsak $x \in X$ enojec $\{x\}$
enak preseku vseh okolic $x$.

$X$ je Hausdorffov natanko tedaj, ko je za vsak $x \in X$ enojec $\{x\}$
enak preseku vseh zaprtih okolic točke $x$.

\vprasanje{Kaj velja za končne $\Ti{1}$ prostore?}

So diskretni.

\vprasanje{Naj bo $X$ regularen prostor. Povej zadosten pogoj, da bo $X$
normalen.}

Biti mora $\Ti{4}$ ali 2-števen.

\vprasanje{Karakteriziraj lastnost $\Ti{3}$.}

Prostor $X$ je $\Ti{3}$ natanko tedaj, ko za vsak $x \in X$ in vsako
okolico $U \ni x$ obstaja manjša okolica $V \ni x$, da je $\cl{V} \subseteq U$.

\vprasanje{Za naslednje lastnosti povej, če se dedne in če so produktne:
diskretnost, trivialnost, 1-števnost, 2-števnost, metrizabilnost.}

Vse so dedne in produktne.

\vprasanje{Kaj velja za dednost separabilnosti in kompaktnosti?}

Separabilnost je dedna na odprte podprostore, kompaktnost pa na zaprte.

\vprasanje{Kaj velja za dednost in produktnost ločljivostnih lastnosti?}

\begin{table}[h!]
\centering
\begin{tabular}{|c|c|c|}
\hline
lastnost       & dedna                 & produktna \\
\hline
Hausdorff      & da                     & da         \\
Freško         & da                     & da         \\
regularnost    & da                     & da         \\
normalnost     & na zaprte podprostore & ne         \\
\hline
\end{tabular}
\end{table}


\vprasanje{Kaj je topologija vsebovane točke?}

Naj bo $X$ množica in $t \in X$. Topologija vsebovane točke je
$\T_{VT} := \{ U \subseteq X \such t \in U \lor U = \varnothing \}$

\vprasanje{Kaj je Sorgenfreyeva premica? Ali je Sorgenfreyeva ravnina $\Ti{4}$?}

To je $\R_S$, opremljena s topologijo, katere baza je
$B_S = \{[a, b) \such a,b \in \R, a < b\}$.

Ni $\Ti{4}$.
